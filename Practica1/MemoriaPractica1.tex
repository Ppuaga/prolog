\documentclass[11pt, a4paper]{article}

\renewcommand{\contentsname}{Contenidos}

% Packages
\usepackage{blindtext}
\usepackage[utf8]{inputenc}
\usepackage{listings}
\usepackage{color}

\definecolor{mygreen}{rgb}{0,0.6,0}
\definecolor{mygray}{rgb}{0.5,0.5,0.5}
\definecolor{mymauve}{rgb}{0.58,0,0.82}

\lstset{ 
  backgroundcolor=\color{white},   % choose the background color; you must add \usepackage{color} or \usepackage{xcolor}; should come as last argument
  basicstyle=\footnotesize,        % the size of the fonts that are used for the code
  breakatwhitespace=false,         % sets if automatic breaks should only happen at whitespace
  breaklines=true,                 % sets automatic line breaking
  captionpos=b,                    % sets the caption-position to bottom
  commentstyle=\color{mygreen},    % comment style
  deletekeywords={...},            % if you want to delete keywords from the given language
  escapeinside={\%*}{*)},          % if you want to add LaTeX within your code
  extendedchars=true,              % lets you use non-ASCII characters; for 8-bits encodings only, does not work with UTF-8
  frame=single,	                   % adds a frame around the code
  keepspaces=true,                 % keeps spaces in text, useful for keeping indentation of code (possibly needs columns=flexible)
  keywordstyle=\color{blue},       % keyword style
  language=prolog,                 % the language of the code
  morekeywords={*,...},            % if you want to add more keywords to the set
  numbers=left,                    % where to put the line-numbers; possible values are (none, left, right)
  numbersep=5pt,                   % how far the line-numbers are from the code
  numberstyle=\tiny\color{mygray}, % the style that is used for the line-numbers
  rulecolor=\color{black},         % if not set, the frame-color may be changed on line-breaks within not-black text (e.g. comments (green here))
  showspaces=false,                % show spaces everywhere adding particular underscores; it overrides 'showstringspaces'
  showstringspaces=false,          % underline spaces within strings only
  showtabs=false,                  % show tabs within strings adding particular underscores
  stepnumber=1,                    % the step between two line-numbers. If it's 1, each line will be numbered
  stringstyle=\color{mymauve},     % string literal style
  tabsize=2,	                   % sets default tabsize to 2 spaces
  title=\lstname                   % show the filename of files included with \lstinputlisting; also try caption instead of title
}

\title{\Huge Memoria Práctica 2 de Programación Declarativa}
\date{Martes 5 de Junio, 2018}
\author{Kiyoshi José Omaza Saldaña\\ v13m080
\and Sergio Fontán Llamas\\ u12m073
\and Pablo Puado García\\ v13m047}


\begin{document}
\pagenumbering{gobble}
\begin{titlepage}
\maketitle
\end{titlepage}

\tableofcontents
\newpage
\pagenumbering{arabic}
\section{cierre, cierreMinimo  y  cierreUnico}
Una fábrica metalúrgica se está haciendo famosa gracias a unas cadenas que produce que han sido diseñadas por ella y resultan especialmente resistentes. Una de las claves del éxito está en los eslabones que emplean, que tienen un cierre especial en cada uno de sus dos extremos. Han diseñado cierres de tipos muy distintos, de forma que dos eslabones se pueden ensamblar entre sí por sus extremos solo si ambos extremos tienen el mismo tipo de cierre.\\\\
El departamento de calidad está sometiendo las cadenas a pruebas de resistencia, para lo que utiliza cadenas cerradas, esto es, con eslabones ensamblados en línea uno tras otro y el último ensamblado con el primero. Para ayudarse en sus pruebas han solicitado un programa informático con las siguientes características. La realización del correspondiente código Prolog es lo que se solicita a los alumnos en esta práctica.\\\\
El predicado $\textbf{cierre/2}$ debe tener éxito cuando su primer argumento es una lista de eslabones sin repeticiones y el segundo una lista que representa una cadena cerrada de eslabones de la primera lista (todos o solo algunos). Solo es necesario que funcione correctamente cuando el primer argumento se da en la llamada.\\\\
Los eslabones se representan con términos $eslabon/2$ cuyos argumentos representan los tipos de cierre en cada uno de sus extremos. Un eslabón con cierres tipo a y b se puede representar como $eslabon(a,b)$ o $eslabon(b,a)$ pero las listas de eslabones (que no tienen repeticiones) solo contendrán uno de ellos. Los tipos de cierres de un mismo eslabón nunca son iguales. Una lista de eslabones que representa una cadena cerrada debe contenerlos en el orden en que ensamblan y el último eslabón debe poder ensamblar con el primero.\\\\
Desgraciadamente, una cadena cerrada se puede representar como lista de eslabones de muchas formas distintas, dependiendo del que se coloque como primer elemento: $[A,B,C,D], [B,C,D,A], [C,D,A,B]$ y $[D,A,B,C]$ representarían la misma cadena cerrada. Esto también ocurre si la cadena se recorre en un sentido o en el otro: $[D,C,B,A]$ también representa la misma cadena cerrada. Algo similar puede ocurrir por otras razones, pero en todos los casos en que ocurre la diferencia entre las listas está en el orden, no en los eslabones que contienen.\\\\
Como lo que interesa de las cadenas cerradas para las pruebas de resistencia son los eslabones que la forman, no el orden en que se ensamblan, esa multiplicidad de listas que representan la misma cadena resulta excesiva. Así que de todas las listas de cadenas cerradas que contengan los mismos elementos pero en distinto orden basta con obtener solo una de ellas. También va a interesar obtener el número mínimo de eslabones necesarios para formar una cadena cerrada.\\\\
El predicado $\textbf{cierreUnico/2}$ debe comportarse igual que $cierre/2$ pero de forma que cada solución que devuelva en su segundo argumento no debe contener los mismos elementos que otra de las soluciones que devuelve. Es decir, en el caso $[A,B,C,D]$ del ejemplo anterior solo debe devolver una de las (ocho) listas posibles. En cambio, el predicado $cierre/2$ debe devolver siempre todas, sin importar que solo difieran en el orden.\\\\
El predicado $\textbf{cierreMinimo/2}$ debe tener éxito cuando su primer argumento es una lista de eslabones sin repeticiones y su segundo argumento es el mínimo número de eslabones que contiene que pueden formar cadena cerrada. En este predicado se valorará positivamente que el código no obtenga todas las soluciones posibles para luego compararlas.\\\\
Cuando se ensamblan físicamente dos eslabones se orientan de manera que los extremos de cada uno que se van a unir estén enfrentados el uno al otro. Sin embargo, esto no ocurre en el programa. Si la lista inicial de eslabones contiene $eslabon(a,b)$, éste se puede utilizar para una cadena cerrada enlazándolo mediante a con el anterior y mediante b con el siguiente, pero también al revés: mediante b con el anterior y mediante a con el siguiente, aunque el término que aparezca en ambos casos en la lista de la cadena cerrada debe ser $eslabon(a,b)$. (Es decir, no se “re-orienta” poniéndolo como $eslabon(b,a)$).\\\\
\subsection{Código utilizado}
\subsubsection{Predicado remove/3}
\begin{lstlisting}[frame=single]
remove(E, [E|T], T).
remove(E, [H|T], [H|TE]):-
  remove(E, T, TE).
\end{lstlisting}
Este predicado elimina el elemento que se le indique de dentro de una lista.
\subsubsection{Predicado belong/2}
\begin{lstlisting}[frame=single]
belong(E, [E|_]).
belong(E, [_|T]):-
  belong(E, T).
\end{lstlisting}
El predicado determina si un elemento pertenece a determinada lista.
\subsubsection{Predicado dolist/3}
\begin{lstlisting}[frame=single]
do_list(N, L):- do_list1(N, [], L).
do_list1(2, L, L):-
  !.
do_list1(N, R, L):-
  N > 0, N1 is N-1, do_list1(N1, [N|R], L).
\end{lstlisting}
Este predicado genera una lista de 3 hasta N elementos.
\subsubsection{Predicado secondlast/2}
\begin{lstlisting}[frame=single]
second_last([X,_], X).
second_last([_|Xs], X):-
  second_last(Xs, X).
\end{lstlisting}
El predicado obtiene el penúltimo elemento de una lista dada.
\subsubsection{Predicado comb/3}
\begin{lstlisting}[frame=single]
comb(0, _Xs, []).
comb(N, [X|Xs], [X|Cs]):-
  N > 0,
  N1 is N-1,
  comb(N1, Xs, Cs).
comb(N, [_X|Xs], Cs):-
  N > 0,
  comb(N, Xs, Cs).
\end{lstlisting}
Este predicado devuelve todas las posibles combinaciones de longitud N de una lista dada.
\subsubsection{Predicado perm/2}
\begin{lstlisting}[frame=single]
perm([], []).
perm([X|Xs], PXs):-
  perm(Xs, Ps),
  remove(X, PXs, Ps).
\end{lstlisting}
Este predicado devuelve todas las posibles permutaciones de una lista dada.
\subsubsection{Predicado combN/3}
\begin{lstlisting}[frame=single]
comb_N(N, L, X):-
  do_list(N, Ln),
  belong(A, Ln),
  apply_CyP(A, L, X).
\end{lstlisting}
Este predicado es una combinación de tres otros predicados (aparecen en el código) y devuelve todas las combinaciones, ya permutadas, de tamaño N de una lista dada.
\subsubsection{Predicado combone/3}
\begin{lstlisting}[frame=single]
comb_one(N, L, X):-
  do_list(N, Ln),
  belong(A, Ln),
  comb(A, L, X).
\end{lstlisting}
Al igual que el predicado anterior, este está formado por otros del código. Y en este caso se obtienen varias combinaciones distintas, cada una de un tamaño distinto, desde tamaño 3 a N.
\subsubsection{Predicado applyCyP/3}
\begin{lstlisting}[frame=single]
apply_CyP(N, L, X):-
  comb(N, L, Xs),
  perm(Xs, X).
\end{lstlisting}
Es una combinación al igual que los anteriores, y en este caso se obtienen todas las posibles combinaciones, ya permutadas, de una lista.
\subsubsection{Predicado islinked/3}
\begin{lstlisting}[frame=single]
is_linked(eslabon(X,_), eslabon(X,_), X).
is_linked(eslabon(X,_), eslabon(_,X), X).
is_linked(eslabon(_,X), eslabon(_,X), X).
is_linked(eslabon(_,X), eslabon(X,_), X).
\end{lstlisting}
Se utiliza para comprobar si dos eslabones están enlazados entre ellos.
\subsubsection{Predicado isband/3}
\begin{lstlisting}[frame=single]
is_band(A, B, C):-
  is_linked(A, B, X),
  is_linked(B, C, Y),
  X\=Y.
\end{lstlisting}
Este predicado comprueba si 3 eslabones dados son adyacentes entre ellos.
\subsubsection{Predicado vertexsame/1}
\begin{lstlisting}[frame=single]
vertex_same([eslabon(X, Y)]):- 
  X==Y.
vertex_same([eslabon(X, Y)|_]):- 
  X==Y.
vertex_same([eslabon(_,_)|Xs]):- 
  vertex_same(Xs).
\end{lstlisting}
Este predicado comprueba si los extremos del eslabón son iguales o distintos.
\subsubsection{Predicado eslabonsame/1}
\begin{lstlisting}[frame=single]
eslabones_same([eslabon(A, B), Y|Xs]):-
  remove(eslabon(A, B), [eslabon(A, B), Y|Xs], [Y|Xs]),
  belong(eslabon(A, B), [Y|Xs]).
eslabones_same([eslabon(A, B), Y|Xs]):-
  remove(eslabon(A,B), [eslabon(A, B), Y|Xs], [Y|Xs]),
  belong(eslabon(B, A), [Y|Xs]).
eslabones_same([_|Xs]):-
  eslabones_same(Xs).
\end{lstlisting}
Este predicado comprueba si los eslabones son iguales $(eslabon(a,b)=eslabon(b,a))$.
\subsubsection{Predicado eslabonall/1}
\begin{lstlisting}[frame=single]
eslabones_all([A, B, C]):-
  is_band(A, B, C),
  is_band(B, C, A),
  is_band(C, A, B).
eslabones_all([A, B, C|Cs]):-
  Cs\=[],
  last([A, B, C|Cs], U),
  second_last([A, B, C|Cs], P),
  is_band(P, U, A),
  is_band(U, A, B),
  right_band([A, B, C|Cs]).	
\end{lstlisting}
Este predicado recibe como argumento una lista.
El predicado devuelve todas las cadenas posibles que se pueden obtener a partir de la lista de eslabones dada.
\subsubsection{Predicado rightband/1}
\begin{lstlisting}[frame=single]
right_band([A, P, U]):-
  is_band(A, P, U).
right_band([A, B, C|Cs]):-
  Cs\=[],
  is_band(A, B, C),
  right_band([B, C|Cs]).
\end{lstlisting}
Este predicado esta hecho para recorrer las cadenas que se van formando.
\subsubsection{Predicado joineslabon/2}
\begin{lstlisting}[frame=single]
join_eslabon(L, X):-
  perm(L, X),
  eslabones_all(X),
  !.
\end{lstlisting}
Este predicado une las cadenas dadas en una misma.
\subsubsection{Predicado joineslabon/2}
\begin{lstlisting}[frame=single]
not(Goal) :-
    call(Goal),
    !,
    fail.
not(_).
\end{lstlisting}
Simple predicado auxiliar para negar un elemento.
\subsubsection{Predicado cierre/2}
\begin{lstlisting}[frame=single]
cierre(Le, X):-
  length(Le, T), T >= 3,
  not(vertex_same(Le)),
  not(eslabones_same(Le)),
  comb_N(T, Le, X),
  eslabones_all(X).
\end{lstlisting}
Este es uno de los tres métodos principales dados por la práctica.
Este predicado recibe una lista de eslabones y devuelve otra lista con todas las posibles cadenas que se pueden formar con los eslabones del primer argumento.
Las cadenas que se obtienen son todas correctas y cumplen los requisitos establecidos por la práctica.
\subsubsection{Predicado cierreMinimo/2}
\begin{lstlisting}[frame=single]
cierreMinimo(L, N):-
  cierre(L, S), !, size(S, N).
\end{lstlisting}
Este es otro de los tres métodos principales dados por la práctica.
Este predicado es una extensión del método $cierre/2$, pero en este caso se devuelve el número mínimo de eslabones de la lista del argumento uno con los que se puede formar una cadena correctamente.
\subsubsection{Predicado cierreUnico/2}
\begin{lstlisting}[frame=single]
cierreUnico(LE, Xs):-
  length(LE, T), T >= 3,
  not(vertex_same(LE)),
  not(eslabones_same(LE)),
  comb_one(T, LE, X),
  join_eslabon(X, Xs).
\end{lstlisting}
Este es otro de los tres métodos principales dados por la práctica.
Al igual que $cierreUnico/2$ es una extensión de $cierre/2$, pero en esta ocasión el predicado devuelve una lista de cadenas en las que los eslabones ya usados en una cadena no van a poder estar en otra cadena.

\subsection{Pruebas realizadas}
\begin{lstlisting}[frame=single]
?- cierre([eslabon(c,b),eslabon(c,d),eslabon(a,b),eslabon(e,b),eslabon(d,b),eslabon(a,e)], Cierre).
Cierre = [eslabon(c,b),eslabon(c,d),eslabon(d,b)] ? n
Cierre = [eslabon(c,d),eslabon(c,b),eslabon(d,b)] ? n
Cierre = [eslabon(c,d),eslabon(d,b),eslabon(c,b)] ? n
Cierre = [eslabon(c,b),eslabon(d,b),eslabon(c,d)] ? n
Cierre = [eslabon(d,b),eslabon(c,b),eslabon(c,d)] ? n
Cierre = [eslabon(d,b),eslabon(c,d),eslabon(c,b)] ? n
Cierre = [eslabon(a,b),eslabon(e,b),eslabon(a,e)] ? n
Cierre = [eslabon(e,b),eslabon(a,b),eslabon(a,e)] ? n
Cierre = [eslabon(e,b),eslabon(a,e),eslabon(a,b)] ? n
Cierre = [eslabon(a,b),eslabon(a,e),eslabon(e,b)] ? n
Cierre = [eslabon(a,e),eslabon(a,b),eslabon(e,b)] ? n
Cierre = [eslabon(a,e),eslabon(e,b),eslabon(a,b)] ? n
Cierre = [eslabon(e,b),eslabon(c,b),eslabon(c,d),eslabon(d,b),eslabon(a,b),eslabon(a,e)] ? n
Cierre = [eslabon(e,b),eslabon(d,b),eslabon(c,d),eslabon(c,b),eslabon(a,b),eslabon(a,e)] ? n
Cierre = [eslabon(a,b),eslabon(c,b),eslabon(c,d),eslabon(d,b),eslabon(e,b),eslabon(a,e)] ? n
Cierre = [eslabon(a,b),eslabon(d,b),eslabon(c,d),eslabon(c,b),eslabon(e,b),eslabon(a,e)] ? n
Cierre = [eslabon(c,b),eslabon(c,d),eslabon(d,b),eslabon(e,b),eslabon(a,e),eslabon(a,b)] ? n
Cierre = [eslabon(c,d),eslabon(d,b),eslabon(e,b),eslabon(a,e),eslabon(a,b),eslabon(c,b)] ? n
Cierre = [eslabon(d,b),eslabon(c,d),eslabon(c,b),eslabon(e,b),eslabon(a,e),eslabon(a,b)] ? n
Cierre = [eslabon(d,b),eslabon(e,b),eslabon(a,e),eslabon(a,b),eslabon(c,b),eslabon(c,d)] ? n
Cierre = [eslabon(c,b),eslabon(c,d),eslabon(d,b),eslabon(a,b),eslabon(a,e),eslabon(e,b)] ? n
Cierre = [eslabon(c,d),eslabon(d,b),eslabon(a,b),eslabon(a,e),eslabon(e,b),eslabon(c,b)] ? n
Cierre = [eslabon(d,b),eslabon(c,d),eslabon(c,b),eslabon(a,b),eslabon(a,e),eslabon(e,b)] ? n
Cierre = [eslabon(d,b),eslabon(a,b),eslabon(a,e),eslabon(e,b),eslabon(c,b),eslabon(c,d)] ? n
Cierre = [eslabon(c,d),eslabon(c,b),eslabon(e,b),eslabon(a,e),eslabon(a,b),eslabon(d,b)] ? n
Cierre = [eslabon(e,b),eslabon(a,e),eslabon(a,b),eslabon(c,b),eslabon(c,d),eslabon(d,b)] ? n
Cierre = [eslabon(c,b),eslabon(e,b),eslabon(a,e),eslabon(a,b),eslabon(d,b),eslabon(c,d)] ? n
Cierre = [eslabon(e,b),eslabon(a,e),eslabon(a,b),eslabon(d,b),eslabon(c,d),eslabon(c,b)] ? n
Cierre = [eslabon(c,d),eslabon(c,b),eslabon(a,b),eslabon(a,e),eslabon(e,b),eslabon(d,b)] ? n
Cierre = [eslabon(a,b),eslabon(a,e),eslabon(e,b),eslabon(c,b),eslabon(c,d),eslabon(d,b)] ? n
Cierre = [eslabon(c,b),eslabon(a,b),eslabon(a,e),eslabon(e,b),eslabon(d,b),eslabon(c,d)] ? n
Cierre = [eslabon(a,b),eslabon(a,e),eslabon(e,b),eslabon(d,b),eslabon(c,d),eslabon(c,b)] ? n
Cierre = [eslabon(a,e),eslabon(e,b),eslabon(c,b),eslabon(c,d),eslabon(d,b),eslabon(a,b)] ? n
Cierre = [eslabon(a,e),eslabon(e,b),eslabon(d,b),eslabon(c,d),eslabon(c,b),eslabon(a,b)] ? n
Cierre = [eslabon(a,e),eslabon(a,b),eslabon(c,b),eslabon(c,d),eslabon(d,b),eslabon(e,b)] ? n
Cierre = [eslabon(a,e),eslabon(a,b),eslabon(d,b),eslabon(c,d),eslabon(c,b),eslabon(e,b)] ? n
no

?- cierre([eslabon(c,b),eslabon(c,d),eslabon(a,b)], Cierre).
no

?- cierre([eslabon(c,b),eslabon(c,d),eslabon(d,b)], Cierre).
Cierre = [eslabon(c,b),eslabon(c,d),eslabon(d,b)] ? n
Cierre = [eslabon(c,d),eslabon(c,b),eslabon(d,b)] ? n
Cierre = [eslabon(c,d),eslabon(d,b),eslabon(c,b)] ? n
Cierre = [eslabon(c,b),eslabon(d,b),eslabon(c,d)] ? n
Cierre = [eslabon(d,b),eslabon(c,b),eslabon(c,d)] ? n
Cierre = [eslabon(d,b),eslabon(c,d),eslabon(c,b)] ? n
no

?- cierre([eslabon(c,b),eslabon(c,d),eslabon(d,b)], [eslabon(c,d),eslabon(d,b),eslabon(c,b)]).
yes

?- cierreMinimo([eslabon(c,b),eslabon(c,d),eslabon(d,b)],X).
X = 3 ? n 
no

?- cierre([eslabon(c,b),eslabon(d,b)],X).
no
\end{lstlisting}
\tiny *Se han eliminado los guiones bajos de algunos de los nombres de los predicados por un problemilla técnico con \LaTeX . En los fragmentos de código insertados aquí y en el archivo .pl si que aparecen ciertos guiones.
\\
Muchas gracias por su comprensión.
\end{document}

